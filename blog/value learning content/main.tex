\documentclass{beamer}
\usefonttheme{professionalfonts}
\usepackage[utf8]{inputenc}
\usepackage{fourier} %font utopia imported
\usepackage[normalem]{ulem}
\usepackage{amssymb}
\usepackage{mathtools}
\usepackage{animate}
\usepackage{ragged2e}
\usepackage[english]{babel}
\usepackage{amsmath,bm}
\usepackage{upgreek}
\usepackage{multicol}
\usepackage{textcomp}
\usepackage[ruled,vlined]{algorithm2e}
\usepackage{bbm}
\usepackage{bigints}
\usepackage{hyperref}
\usepackage{multimedia}
\usepackage{dsfont}
\usepackage[export]{adjustbox}
\usepackage[backend=biber,style=numeric,sortcites,sorting=none,backref,natbib,hyperref]{biblatex}
\bibliography{references}
\usetheme{Madrid}
\usecolortheme{default}

\DeclareMathOperator{\E}{\mathbb{E}}
\setbeamertemplate{footline}[frame number]
\newcommand{\bigCI}{\mathrel{\text{\scalebox{1.07}{$\perp\mkern-10mu\perp$}}}}

\newcommand{\R}{{\mathbb R}}
\def\vec#1{\mathbf{#1}}
\def\dt#1{\dot{#1}}

%------------------------------------------------------------
%This block of code defines the information to appear in the
%Title page
\title %optional
{Optimal Control Theoretic Value Function Learning}


% \logo{\includegraphics[height=1.5cm]{lion-logo.jpg}}

%End of title page configuration block
%------------------------------------------------------------



%------------------------------------------------------------
%The next block of commands puts the table of contents at the 
%beginning of each section and highlights the current section:

% \AtBeginSection[]
% {
%   \begin{frame}
%     \frametitle{Table of Contents}
%     \tableofcontents[currentsection]
%   \end{frame}
% }
%------------------------------------------------------------


\begin{document}

%The next statement creates the title page.
\frame{\titlepage}


%---------------------------------------------------------
%This block of code is for the table of contents after
%the title page

%---------------------------------------------------------


\section{Problem Statement}

%---------------------------------------------------------

\begin{frame}
\frametitle{Problem Statement}
Typical tasks in robotics
% \begin{figure}
%     \includegraphics[width=.5\textwidth,center]{Screenshot from 2021-10-26 18-29-19.png}
% \end{figure}

\begin{columns}[T] % align columns
    \begin{column}{.5\textwidth}
        \begin{figure}
            \centering
            \includegraphics[width=.6\textwidth]{Screenshot from 2021-10-26 18-29-19.png}
        \end{figure}
    \end{column}%
    \hfill%
    \begin{column}{.5\textwidth}
        \begin{figure}
            \centering
            \includegraphics[width=.7\textwidth]{Screenshot from 2021-10-26 18-39-19.png}
        \end{figure}
    \end{column}%
\end{columns}

\begin{columns}[T] % align columns
    \begin{column}{.5\textwidth}
        \begin{figure}
            \centering
            \includegraphics[width=.6\textwidth]{Screenshot from 2021-10-26 18-31-36.png}
        \end{figure}
    \end{column}%
    \hfill%
    \begin{column}{.5\textwidth}
        \begin{figure}
            \centering
            \includegraphics[width=.6\textwidth]{hand.jpeg}
        \end{figure}
    \end{column}%
\end{columns}
\end{frame}


%---------------------------------------------------------
\begin{frame}
\frametitle{Tasks and Objective Landscapes}
What is the commonality between them?
\begin{figure}%[2]
    \centering
    \includegraphics[scale=0.12]{Screenshot from 2021-10-26 17-39-53.png}
\end{figure}
\begin{itemize}
    \item Generally, objective landscapes are flat.
    \item Not suitable for optimisation 
\end{itemize}

\end{frame}

\begin{frame}
\frametitle{General Approaches}
  \begin{equation}
    \text{Approaches to decision making ~}
    \begin{cases}
      \textbf{OC} \\
      \textbf{RL}
    \end{cases}
    \nonumber
  \end{equation}
  \\~\\
  
\begin{columns}[T] % align columns
\begin{column}{.48\textwidth}
\color{red} OC\\ \rule{\linewidth}{4pt}
\color{black}
\begin{itemize}
    \item DDP, ILQR, Collocation\\
        \begin{itemize}
            \item local
            \item \textbf{efficient} near optima
            \item strict objective constraints
        \end{itemize}
    \item \text{MPPI, CEM, RS}\\
    \begin{itemize}
        \item local
        \item cost of sampling
        \item \textbf{flexible} objective
    \end{itemize}
\end{itemize}
\end{column}%
\hfill%
\begin{column}{.48\textwidth}
\color{blue} RL \\\rule{\linewidth}{4pt}
\color{black}
\begin{itemize}
    \item global and \textbf{general}\\
    \item \textbf{flexible} objective \\
    \item Inherent exploration \\
    \item slow convergence
\end{itemize}
\end{column}%
\end{columns}
  
% The numerical approach to optimal control is highly efficient in solving for locally optimal controls in \textbf{well defined} scenarios. 

% \begin{alertblock}{Requirements}
% \begin{itemize}
%     \item Continuous~dynamics\\
%     \item Cost~is~not~sparse\\
%     \item Differentiable
% \end{itemize}
% \end{alertblock}
\end{frame}

% \begin{frame}{Problem Statement}

% Inference approach to optimal control have some benefits:

% \begin{exampleblock}{For example}
% \begin{itemize}
%     \item Tolerance to non-convexity\\
%     \item Tolerance  to  non-existent derivatives (i.e., discontinuities)\\
%     \item Ability to explore due to the assumption of input noise
% \end{itemize}
% \end{exampleblock}

% However, inferring the unknown distribution is computationally costly.
    
% \end{frame}

%---------------------------------------------------------

\begin{frame}
  \centering % Centers everything on the page
  \vfill % Adds vertical space before the content
  \begin{beamercolorbox}[sep=8pt,center]{title}
    \usebeamerfont{title}Flexibility + Efficiency
  \end{beamercolorbox}
  \vfill % Adds vertical space after the content
  \vfill \tiny {Optimal Control via Combined Inference and Numerical Optimization (ICRA22)}
\end{frame}

\begin{frame}
\frametitle{Problem statement}
Numerical optimisation is efficient at refinement but fails to \textbf{discover} control policies or behaviours far away from the point of initialisation. Sampling is able to discover further local optima through exploration with simpler cost functions.
\\~\\
\begin{exampleblock}{Aim}
Combine the strength of both inference and second-order trajectory optimization.  We aim to formulate a natural combination that can  \textbf{efficiently sample}  when  \textit{no derivative information}  is available and follow  \textbf{optimized trajectories} when \textit{derivatives arise}.
\end{exampleblock}
\end{frame}

\begin{frame}
\frametitle{Numerical solution}
The method relies on the Bellman equation and its tangent space solution \cite{tassa2012synthesis}.
\begin{equation}\label{eq: bellman}
v(\mathbf{x}, t) = \min_{\mathbf{u}} {\underbrace{\left[\ell(\mathbf{x}, \mathbf{u}, t) + v\left({\mathbf{f}(\mathbf{x}, \mathbf{u}, t)}\right) \right]}_{Q(\mathbf{x}, \mathbf{u}, t)}}  \nonumber
\end{equation}

Solving in tangent space
\begin{equation}
\begin{aligned}
Q(\delta{\mathbf{x}}, \delta{\mathbf{u}}) &\approx \frac{1}{2}{\begin{bmatrix}1 \\ \delta{\mathbf{x}} \\ \delta{\mathbf{u}} \end{bmatrix}}^{\top}\begin{bmatrix} 0 & Q_{\mathbf{x}}^{\top} & Q_{\mathbf{u}}^{\top} \\
Q_{\mathbf{x}} & Q_{\mathbf{xx}} & Q_{\mathbf{xu}} \\ 
Q_\mathbf{u} & Q_{\mathbf{ux}} & Q_{\mathbf{uu}} \end{bmatrix} \begin{bmatrix} 1 \\\delta{\mathbf{x}} \\ \delta{\mathbf{u}} \end{bmatrix} \nonumber
\end{aligned}
\end{equation}

\begin{exampleblock}{Update Law}
This leads to an updated law similar to Newton's method:
\begin{equation}
\hat{\mathbf{u}} = \mathbf{u} - ({Q_\mathbf{u}} - Q_{\mathbf{ux}}\delta{\mathbf{x}})Q^{-1}_{\mathbf{uu}}
\nonumber
\end{equation}
\end{exampleblock}
\vfill \tiny {Differential Dynamic Programming–A Unified Approach to the Optimization of Dynamic Systems. Mayne}
\end{frame}

% %---------------------------------------------------------

\begin{frame}
\frametitle{Inference solution}

The KL control approach is grounded in the stochastic version of the Bellman equation \cite{theodorou2012relative}.

\begin{equation}
\begin{aligned}
\underset{\mathbf{u}}{\text{min}}[\ell(\mathbf{x}) + &\text{KL}(\mathbb{Q}||\mathbb{P}) + \E_{\mathbb{Q}}[v(\mathbf{x}')]]\\
\text{KL}(\mathbb{Q}||\mathbb{P}) &= \E_{\mathbb{Q}}\left[\log\left[\frac{\mathbf{p}(\mathbf{x}'|\mathbf{x}, \mathbf{u})}{\mathbf{p}(\mathbf{x}'|\mathbf{x})}\right]\right]
\nonumber
\end{aligned}
\end{equation}
Where $\mathbf p(\mathbf{x}'|\mathbf{x}, \mathbf{u})$ and $\mathbf p(\mathbf{x}'|\mathbf{x})$ represent the controlled and uncontrolled probabilities

\begin{exampleblock}{Optimal controls}
Leads to an optimal control distribution:
\begin{equation}
\mathbf{p}^*(\mathbf{x}'|\mathbf{x}, \mathbf{u}) = \mathbf{p}(\mathbf{x}'|\mathbf{x})\text{exp}\left(-v(\mathbf{x}')\right)\eta^{-1}
\nonumber
\end{equation}
\end{exampleblock}
\vfill \tiny {Information-theoretic model predictive control: Theory and applications to autonomous driving. Williams et al.}
\end{frame}

%------------------------------------------------------------
\begin{frame}
\frametitle{Combining inference with second order method}
Reformulate the stochastic Bellman equation and add a secondary divergence term with the solution of the Bellman equation in the tangent space.

\begin{equation}
\begin{aligned}\label{eq: combined Bellman}
\underset{\mathbf{u}}{\text{min}}[l(\mathbf{x}) + (1-k)&\text{KL}(\mathbb{Q}||\mathbb{P}) + k \text{KL}(\mathbb{Q}||\mathbb{C}) + \E_{\mathbb{Q}}[v(\mathbf{x}')]]\\
\text{KL}(\mathbb{Q}||\mathbb{C})& = \E_{\mathbb{Q}}\left[\log\left[\frac{\mathbf{p}(\mathbf{x}'|\mathbf{x}, \mathbf{u})}{\mathbf{p}(\mathbf{x}'|\mathbf{x}, \mathbf{u}_{\text{iLQG}})}\right]\right]
\nonumber
\end{aligned}
\end{equation}

\begin{exampleblock}{Update Law}
Using a sequence of control trajectories $\mathbf W = [\mathbf{u}_{0}, ..., \mathbf{u}_{T-1}]$ sampled from either controlled or uncontrolled distributions, we can represent the optimal distribution over control trajectories:
\begin{equation}
\mathbf{q}^*(\mathbf W) = \mathbf{p}(\mathbf W)^{1-k}\mathbf{c}(\mathbf W)^{k}\text{exp}\big(-\frac{1}{\lambda}S(\mathbf W)\big)\eta^{-1}
\nonumber
\end{equation}
\begin{center}
$\mathbf{p}(\mathbf W) := \mathbf{w}_t \sim \mathcal{N}(0, \mathbf{\Sigma}),~\mathbf{q}(\mathbf W) := \mathbf{w}_t \sim \mathcal{N}(\mathbf{u}_t, \mathbf{\Sigma}),~\mathbf{c}(\mathbf W) := \mathbf{w}_t \sim \mathcal{N}(\mathbf{u}_{\text{iLQG}_t},\mathbf{\Sigma}_{\text{iLQG}})$
\end{center}
\end{exampleblock}
\end{frame}


\begin{frame}
\frametitle{Importance sampling}
We cannot directly sample from $\mathbf{q}^*(\vec W)$ we importance sample:

\begin{equation}\label{eq: importance sampling}
\mathbf{u}^* = \bigintsss\mathbf{q}(\vec W){w(\vec W)} v_t \mathrm{d} W
\nonumber
\end{equation}

Where the sample weights become:

\begin{equation}
\begin{aligned}\label{eq: sample weights}
w(\vec W) = \frac{1}{\eta}\exp\Big(-\frac{1}{2}\frac{1}{\lambda}S(\vec W)+\sum_{t=0}^{T} k\big({(\mathbf{\vec w}_{t}-\mathbf{u}_{\text{iLQG}_t})}^\top\mathbf{\Sigma}_{\text{iLQG}_t}^{-1}(\mathbf{\vec w}_{t}-\mathbf{u}_{\text{iLQG}_t})...\big)\Big)
\end{aligned}
\nonumber
\end{equation}

\begin{exampleblock}{Update Law}
iLQG distribution found by maximising $\ln \exp (-\delta Q_t(\mathbf{x}_t, \mathbf{u}_t))$. Samples from $\mathbb{C}$ take the form $\vec w_t \sim \mathcal{N}(\mathbf{u}_{\text{iLQG}_t}, Q_{\mathbf{uu}_t}^{-1})$
\end{exampleblock}
\end{frame}

% \begin{frame}
% \frametitle{Final Algorithm}
% \fontsize{10pt}{10}\selectfont
% \begin{center}
% \begin{algorithm}[H]
% \SetAlgoLined
% $K$: \text{Number of samples} and $T$: \text{Number of time-steps}\\
% $q_{\text{iLQG}_r}(\textbf{x}_t)$ and $q_{r}(\textbf{x}_t)$: \text{Running costs}\\
% $q_{\text{iLQG}_t}(\textbf{x}_T)$ and $q_{t}(\textbf{x}_T)$: \text{Terminal costs}\\
% $\textbf{u}_{0:T-1}$ and $\textbf{u}_{\text{iLQG}_{0:T-1}}$: Initial trajectories\\
% $\lambda,~k,~\beta,~\mathbf{\Sigma}$: Hyper parameters for eq. \ref{eq: sample weights}\\
% \While{not done}
% {
%     $\mathbf{x}_0$ = EstimateState()\\
%     $[\textbf{u}_{\text{iLQG}~0:T-1}, ~Q_{\mathbf{uu}~0:T-1}]$ = iLQG($\textbf{x}_0$)\\
%     $\mathbf{\Sigma}_{\text{iLQG}~0:T-1} = \beta ~Q^{-1}_{\mathbf{uu}~0:T-1}$\\
%     $\textbf{u}_{0:T-1}$ = KLControl($\textbf{u}_{\text{iLQG}~0:T-1},~\mathbf{\Sigma}_{\text{iLQG}~0:T-1},~\textbf{x}_0$) MPPI with updated weights\\
%     apply $\textbf{u}_0$\\
%     $\textbf{u}_{0:T-1} = \textbf{u}_{1:T-1}$\\
%     $\textbf{u}_{\text{iLQG}~0:T-1} =\textbf{u}_{0:T-1}$
% }
% \caption{Inference with Trajectory Optimization (Ours)}
% \label{alg: InTO}
% \end{algorithm}
% \end{center}
% \end{frame}


\begin{frame}
\frametitle{Results}
\fontsize{8pt}{10}\selectfont
\begin{center}
\movie[width=7cm,height=5.5cm,poster,showcontrols]{Click to view}{ICRA.mp4}    
\end{center}
\end{frame}


\begin{frame}
\frametitle{Conclusion}

\begin{exampleblock}{Contribution}
\begin{itemize}
    \item Natural combination inference and numerical optimisation.
    \item Inherent ability to explore. 
    \item Combine convex and non-convex costs.
\end{itemize}
\end{exampleblock}

\begin{alertblock}{Caveats}
\begin{itemize}
    \item Search across hyperparamets: $\lambda$, $\beta$ and $k$.
    \item Is $Q_{uu}^{-1}$ not necessarily a good measure of confidence.  
    \item Constant control input noise affects convergence.
    \item In the worst case, it can perform just as badly as the other methods.
\end{itemize}
\end{alertblock}
\vfill \tiny {Optimal Control via Combined Inference and Numerical Optimization (ICRA22)}
\end{frame}

\begin{frame}
  \centering % Centers everything on the page
  \vfill % Adds vertical space before the content
  \begin{beamercolorbox}[sep=8pt,center]{title}
    \usebeamerfont{title} Generalisation + Efficiency 
  \end{beamercolorbox}
  \vfill % Adds vertical space after the content
  \vfill \tiny {Neural Lyapunov and Optimal Control (in submission)}
\end{frame}

\begin{frame}
\frametitle{Problem statement}
Numerical optimal control solvers parameterise trajectories to provide \textit{locally optimal solutions} given smooth and differentiable objective functions. Conversely, RL parameterises the function space to synthesise \textit{approximate global policies}.
\\~\\
\begin{exampleblock}{Aim}
Combine the generalisation capabilities of RL with the efficiency of OC theoretic solutions. We aim to formulate an appropriate global method that leverages the differentiation operation to enable \textbf{efficiency} inspired by OC and parameterises function spaces in order to achieve \textbf{generalisation}. 
\end{exampleblock}
\end{frame}

\begin{frame}
\frametitle{Optimisation in RL}
\begin{center}
\includegraphics[scale=0.25]{0thvsfirst.png}
\end{center}
\vfill \tiny {The many faces of integration by parts ii: Randomized smoothing and score functions. Bach}
\end{frame}

\begin{frame}
\frametitle{Famous RL environments}
RL environments tinker with dynamics tuning under the hood. 
\begin{center}
\movie[width=9cm,height=7.5cm,poster,showcontrols]{Click to view}{rl_hacks.mp4}    
\end{center}
\end{frame}


\begin{frame}
\frametitle{Continuous Bellman equation}
Our approach relies on the Hamilton-Jacobi-Bellman equation.
\begin{equation}
\label{eq: hjb policy}
-\frac{\partial v}{\partial t}\left(\vec{x}_t, t\right) = \operatorname*{min}_{\vec u \in \vec U} \left [\ell\left(\vec{x}_t, \vec{u}_t\right) + \vec{f}\left(\vec{x}_t, \vec{u}_t\right)^\top \nabla_{\vec x_t}v\left(\vec{x}_t, t\right)^\top\right ]
\nonumber
\end{equation}

\begin{exampleblock}{Optimal controls}
Assuming affine control $\dot {\vec x}_t = \vec h(\vec x_t) + \vec g(\vec x_t) \vec u_t$ and quadratic control cost, we can analytically solve for $\vec u_t$
\begin{equation}
\label{eq: det opt u general}
    \vec u^*(\vec x_t) = -(\nabla^2_{{\vec u}_t} \ell_{\text{ctrl}}(\vec u_t))^{-1}\left( \nabla_{{\vec x}_t} v(\vec x_t, t)\nabla_{{\vec u}_t} \vec f(\vec x_t, u_t)\right)
\nonumber
\end{equation}
\end{exampleblock}

We can compute the optimal greedy control given the optimal value function $v$. This makes our choice of parameterisation clear. \textbf{The value function}.
\end{frame}

\begin{frame}{Local solutions}
\textbf{Pontryagin's minimum principle} is the age-old solution to this problem and the building block to methods like \textit{DDP} and \textit{iLQR} \cite{aceituno2017simultaneous}.
\begin{exampleblock}{Approach}
Move to trajectories space and treat $\nabla_{\vec x_t}v(\vec x_t, t)^\top = \vec p(t)$. The value function is no longer a function of state but is defined by the trajectory the system follows.     
\end{exampleblock}

\begin{block}{Minimum principle}
\begin{columns}
    \begin{column}{0.7\textwidth}
        \begin{equation*}
        \begin{aligned}
            \dot{\vec{x}} &= \vec{f}(\vec{x}_t, \vec{u}_t) \\
            \dot{\vec{p}} &= -\nabla_{\vec{x}_t} \ell(\vec{x}_t, \vec{u}_t) - \nabla_{\vec{x}_t} \vec{f}^\top (\vec{x}_t, \vec{u}_t) \vec{p}(t) \\
            \vec{u}^* &= -(\nabla^2_{{\vec{u}_t}} \ell_{\text{ctrl}}(\vec{u}_t))^{-1}\left(\vec{g}(\vec{x}_t)^\top  \vec p(t)\right)
        \end{aligned}
        \end{equation*}
    \end{column}
    \begin{column}{0.3\textwidth}
        \textbf{Initial} \\
        \(\vec{x} = \vec{x}_0 \in \mathbf{R}^{m}\) \\
        \textbf{Terminal} \\
        \(\vec{p}(T) = \nabla_{\vec{x}_T} \psi(\vec{x}_T)^\top\)
    \end{column}
\end{columns}
\end{block}
\vfill \tiny {Calculus of variations and optimal control theory: a concise introduction. Liberzon}
\end{frame}

\begin{frame}{PMP}
Solve one ODE forward for $\vec{x}$(t) and one backward for ${\vec{p}}$(t) 
\begin{figure}%[2]
    \centering
    \includegraphics[scale=1]{pmp.pdf}
\end{figure}
\end{frame}

\begin{frame}
\frametitle{Making PMP global}
Rearranging the HJB gives us a concise \textbf{constraint on the rate of change of the value function}.
\begin{equation}
\begin{aligned}
\frac{\partial v}{\partial t}\left(\vec{x}_t, t\right)  \vec{f}\left(\vec{x}_t, \vec{u}_t\right)^\top \nabla_{\vec x_t}v\left(\vec{x}_t, t\right) &=-\ell(\vec x_t, \vec u^*_t)\\
    \frac{\partial v(\vec x_t, t)}{\partial t} + \frac{\partial v(\vec x_t, t)}{\partial \vec x_t}\frac{d\vec x_t}{dt} &= -\ell(\vec x_t, \vec u^*_t)\\
    \frac{dv(\vec x_t, t)}{dt} &= -\ell(\vec x_t, \vec u^*_t)\nonumber
\end{aligned}
\end{equation}

\begin{exampleblock}{Intuition}
Under the policy $\vec u^*_t$, the value function decreases at a rate negative to the cost function. 
\end{exampleblock}
\end{frame}

\begin{frame}
\frametitle{Making PMP global}
We can directly \textbf{parameterise the value function} and learn an approximate global value function and, therefore, policy.

\begin{block}{Maximum principle}
\begin{columns}
    \begin{column}{0.7\textwidth}
        \begin{equation*}
        \begin{aligned}
            \dot{\vec{x}} &= \vec{f}(\vec{x}_t, \vec{u}_t) \\
            \dot{v} &= -\ell(\vec{x}_t, \vec{u}_t) \\
            \vec{u}^* &= -(\nabla^2_{{\vec{u}_t}} \ell_{\text{ctrl}}(\vec{u}_t))^{-1}\left(\vec{g}(\vec{x}_t)^\top  \nabla_{{\vec{x}_t}}v(\vec{x}_t, t; \theta)\right)
        \end{aligned}
        \end{equation*}
    \end{column}
    \begin{column}{0.3\textwidth}
        \textbf{Initial} \\
        \(\vec{x} = \vec{x}_0 \in \mathbb{R}^{B\times m}\) \\
        \textbf{Terminal} \\
        \(v(\vec{x}_T, T; \theta) = \psi(\vec{x}_T)\)
    \end{column}
\end{columns}
\end{block}
\end{frame}

\begin{frame}
\frametitle{Dynamic programming view}
The discrete solution of the previous ODE resembles the dynamic programming update.
\begin{align*}
    v(\vec x_{N}, N) &= \psi(\vec x_{N}) \\
    v(\vec x_{N-1}, N-1) &= v(\vec x_{N}, N) + \ell(\vec x_{N-1}, \vec u^*_{N-1}) \times \Delta t \\
    v(\vec x_{N-2}, N-2) &= v(\vec x_{N-1}, N-1) + \ell(\vec x_{N-2}, \vec u^*_{N-2}) \times \Delta t \\
    &\vdots \\
    v(\vec x_{0}, 0) &= v(\vec x_{1}, 1) + \ell(\vec x_{0}, \vec u^*_{0}) \times \Delta t
\end{align*}

\begin{exampleblock}{Intuition}
This is essentially a cumulative sum of costs backwards. 
\end{exampleblock}
\end{frame}


\begin{frame}{Global PMP}
Solve one ODE forward for $\vec x(t)$ and backwards for $v(\vec x(t), t)$
\begin{figure}%[2]
    \centering
    \includegraphics[scale=1]{value.pdf}
\end{figure}\end{frame}

\begin{frame}
\frametitle{Discrete algorithm}
\begin{algorithm}[H]
\SetAlgoLined
\textbf{Initialize:} $\vec{x}(0) = \vec{x}_0$, $\Delta t$, $N = \frac{T}{\Delta t}$, $v(\vec x_N, N) = \psi(\vec{x}_N)$\;
\For{$i = 0$ \KwTo $N-1$}{
    $\vec{u}_i = -(\nabla^2_{\vec{u}_i} \ell_{\text{ctrl}}(\vec{u}_i))^{-1} \vec{g}(\vec{x}_i)^\top\nabla_{\vec{x}} \tilde v(\vec{x}_i, i; \theta)^\top$\;
    $\vec{x}_{i+1} = \vec{x}_i + \vec{f}(\vec{x}_i, \vec{u}_i) \times \Delta t$\;
}
\For{$i = N-1$ \KwTo $0$}{
    $v(\vec x_{i}, i) = v(\vec x_{i+1}, i+1) + \ell(\vec{x}_{i}, \vec{u}_{i}) \times \Delta t$\;
}

\textbf{Optimize:} $\min_\theta \sum_{i=0}^{N} \left(v(\vec x_{i}, i) - \tilde{v}(\vec{x}_i, i, \theta)\right)^2$\;
\caption{Neural OC}
\end{algorithm}
\begin{exampleblock}{Intuition}
    This can be viewed as a form of \textbf{fitted value iteration}. E.g. rollout, compute the values, fit to targets.
\end{exampleblock}
\end{frame}


\begin{frame}
  \frametitle{Results}

  % First row of images
  \begin{columns}
    \column{0.33\textwidth}
    \centering
    \includegraphics[width=\linewidth]{cartpole_balance_trajectory_cost_ieee.png}
    \column{0.33\textwidth}
    \centering
    \includegraphics[width=\linewidth]{cartpole_swingup_trajectory_cost_ieee.png}
    \column{0.33\textwidth}
    \centering
    \includegraphics[width=\linewidth]{reacher_trajectory_cost_ieee.png}
  \end{columns}
  \vspace{0.025cm} % Space between the rows

  % Second row of images
  \begin{columns}
    \column{0.33\textwidth}
    \centering
    \includegraphics[width=\linewidth]{cartpole_balancing_rewards_ieee.png}
    \column{0.33\textwidth}
    \centering
    \includegraphics[width=\linewidth]{cartpole_swingup_rewards_ieee.png}
    \column{0.33\textwidth}
    \centering
    \includegraphics[width=\linewidth]{reahcer_rewards_ieee.png}
  \end{columns}

  \begin{exampleblock}{Overview}
    Significantly faster convergence and variance in
    random seeds. Outperforming SAC and PPO by at least
    a factor of 74 and 2, respectively.
  \end{exampleblock}
\end{frame}
\begin{frame}{Conclusion}
\begin{exampleblock}{Contribution}
\begin{itemize}
    \item Global policy.
    \item Efficiency from direct derivatives
    \item No backprop through time
\end{itemize}
\end{exampleblock}

\begin{alertblock}{Caveats}
\begin{itemize}
    \item Requires full dynamics information.  
    \item No ability to introduce smoothing through noise.
    \item Only quadratic control constraints and no state constraints
\end{itemize}
\end{alertblock}
\vfill \tiny {Neural Lyapunov and Optimal Control (in submission)}
\end{frame}

\begin{frame}
  \centering % Centers everything on the page
  \vfill % Adds vertical space before the content
  \begin{beamercolorbox}[sep=8pt,center]{title}
    \usebeamerfont{title} Generalisation + Efficiency + Flexibility 
  \end{beamercolorbox}
  \vfill % Adds vertical space after the content
  \vfill \tiny {Neural Stochastic Optimal Control (in submission)}
\end{frame}
\begin{frame}
\frametitle{Problem statement}
Numerical optimal control solvers efficiently compute locally optimal trajectories given smooth and differentiable objective functions. Conversely, RL approximates general policies while leveraging sampling to impose minimal constraints on the objective.
\\~\\
\begin{exampleblock}{Aim} 
We aim to formulate an appropriate global method that leverages sampling to enable \textbf{flexibilty} inspired by RL while maintaining \textbf{efficiency} and \textbf{generalisation}. 
\end{exampleblock}
\end{frame}

\begin{frame}
\frametitle{Continuous stochastic bellman equation}
In this case, we use the stochastic HJB equation.
\begin{equation}
\label{eq: stochastic hjb intro}
-\frac{\partial v(\vec x_t, t)}{\partial t}  = \min_{\vec u_t}\left[\ell(\vec x_t, \vec u_t) + \nabla_{\vec{x}_t} v(\vec x_t, t) \vec f(\vec x_t, \vec u_t) + \frac{1}{2} \mathrm{trace}\left(\nabla^2_{\vec{x}_t \vec{x}_t} v(\vec x_t, t)\vec \Sigma(\vec x_t)\right)\right] \nonumber
\end{equation}

\begin{exampleblock}{Optimal controls}
Assuming affine control $d\vec x_t = \left(\vec{h}\left(\vec{x}_t\right) + \vec g(\vec x_t) \vec u_t\right){d}t + \vec{B}\left(\vec{x}_t\right) {d}\vec{w}_t$ and quadratic control cost, we can analytically solve for $\vec u_t$
\begin{equation}
\label{eq: det opt u general}
    \vec u^*(\vec x_t) = -(\nabla^2_{{\vec u}_t} \ell_{\text{ctrl}}(\vec u_t))^{-1}\left( \nabla_{{\vec x}_t} v(\vec x_t, t)\nabla_{{\vec u}_t} \vec f(\vec x_t, u_t)\right)\nonumber
\nonumber
\end{equation}
\end{exampleblock}

Our optimal control remains the same, and so does our choice of parameterisation, \textbf{the value function}.
\end{frame}

\begin{frame}
\frametitle{Learning the value function}
Similar to the deterministic case, we need to define \textbf{constraint on the rate of change of the value function}.
\begin{equation}
\begin{aligned}
\label{eq: expected derivative value with trace}
    \mathbb E_\mathcal Q \left[\frac{dv(\vec x_{t}, t)}{dt}\right] &= -\ell(\vec x_t, \vec u^*_t)\\
    \mathbb E_\mathcal Q \left[v(\vec x_{t}, t) - v(\vec x_{t+\Delta t}, t+\Delta t)\right] &= \ell(\vec x_t, \vec u^*_t) \times \Delta t \\
    v(\vec x_{t}, t) - \mathbb E_\mathcal Q \left[v(\vec x_{t+\Delta t}, t+\Delta t)\right] &= \ell(\vec x_t, \vec u^*_t) \times \Delta t
\end{aligned}
\end{equation}

\begin{exampleblock}{Intuition}
Under the policy $\vec u^*_t$, the expected change of the value function decreases at a rate negative to the cost function. 
\end{exampleblock}
\end{frame}

\begin{frame}
\frametitle{TD(ish) view}
Satisfying the previous constraint over horizon N can be viewed through a lens similar to TD.
\begin{align*}
    \mathbb E_\mathcal Q \left[v(\vec x_{N}, N)\right] -  \mathbb E_\mathcal Q \left[\psi(\vec x_{N})\right] &= 0  \\
    v(\vec x_{N-1}, N-1) - \mathbb E_\mathcal Q \left[v(\vec x_{N}, N)\right] - \ell(\vec x_{N-1}, \vec u^*_{N-1}) \times \Delta t &= 0 \\
    v(\vec x_{N-2}, N-2) - \mathbb E_\mathcal Q \left[v(\vec x_{N-1}, N-1)\right] - \ell(\vec x_{N-2}, \vec u^*_{N-2}) \times \Delta t &= 0 \\
    &\vdots \\
    v(\vec x_{0}, 0) - \mathbb E_\mathcal Q \left[v(\vec x_{1}, 1)\right] - \ell(\vec x_{0}, \vec u^*_{0}) \times \Delta t &= 0
\end{align*}

\begin{exampleblock}{Note}
We cannot simply sum the costs backwards due to the presence of expectation under $\mathcal Q$ 
\end{exampleblock}
\end{frame}

\begin{frame}
\frametitle{Discrete algorithm}
\begin{algorithm}[H]
\SetAlgoLined
\textbf{Initialize:} $\vec{x}(0) = \vec{x}_0$, $\Delta t$, $N = \frac{T}{\Delta t}$, $E_\mathcal Q \left[v(\vec x_N, N)\right] = E_\mathcal Q \left[\psi(\vec{x}_N)\right]$\;
\For{$i = 0$ \KwTo $N-1$}{
    $\vec{u}_i = -(\nabla^2_{\vec{u}_i} \ell_{\text{ctrl}}(\vec{u}_i))^{-1} \vec{g}(\vec{x}_i)^\top\nabla_{\vec{x}} \tilde{v}(\vec{x}_i, i; \theta)^\top$\;
    $\vec{x}_{i+1} = \vec x_t + \left( \vec h(\vec{x}_t) + \mathbf{g}(\vec{x}_t, t)\vec{u}_t \right) \Delta t + \mathbf{B}(\vec{x}_t, t)\vec{\epsilon}\sqrt{\Delta t}$\;
}
\textbf{Optimize:} $\min_\theta \sum_{i=0}^{N} \left(\tilde{v}(\vec x_{i}, i, \theta) - E_\mathcal Q \left[\tilde{v}(\vec{x}_i, i, \theta)\right] - \ell(\vec x_i, \vec u^*_i) \times \Delta t\right)^2$\;
\caption{Neural SOC}
\end{algorithm}
\begin{exampleblock}{Intuition}
Rollout trajectories under the current optimal policy then fit the stochastic Bellman difference. 
\end{exampleblock}
\end{frame}

\begin{frame}
  \frametitle{Results}

  % First row of images
  \begin{columns}
    \column{0.33\textwidth}
    \centering
    \includegraphics[width=\linewidth]{cartpole_balance_trajectory_cost_ieee_stochastic.png}
    \column{0.33\textwidth}
    \centering
    \includegraphics[width=\linewidth]{cartpole_swingup_trajectory_cost_ieee_stochastic.png}
    \column{0.33\textwidth}
    \centering
    \includegraphics[width=\linewidth]{reacher_trajectory_cost_ieee_stochastic.png}
  \end{columns}
  \vspace{0.025cm} % Space between the rows

  % Second row of images
  \begin{columns}
    \column{0.33\textwidth}
    \centering
    \includegraphics[width=\linewidth]{cartpole_balancing_rewards_ieee_stochastic.png}
    \column{0.33\textwidth}
    \centering
    \includegraphics[width=\linewidth]{cartpole_swingup_rewards_ieee_stochastic.png}
    \column{0.33\textwidth}
    \centering
    \includegraphics[width=\linewidth]{reahcer_rewards_ieee_stochastic.png}
  \end{columns}

  \begin{exampleblock}{Overview}
    Significantly faster convergence and variance in
    random seeds. Outperforming SAC and PPO by at least
    a factor of 18 and 2, respectively.
  \end{exampleblock}
\end{frame}


\begin{frame}
  \frametitle{Results}
    Value approximation given stochastic dynamics and discontinuous objective. 

  % First row of images
  \begin{columns}
    \column{0.33\textwidth}
    \centering
    \includegraphics[width=\linewidth]{1stnoise.png}
    \column{0.33\textwidth}
    \centering
    \includegraphics[width=\linewidth]{3rdnoise.png}
    \column{0.33\textwidth}
    \centering
    \includegraphics[width=\linewidth]{4thnoise.png}
  \end{columns}
  \vspace{0.025cm} % Space between the rows
  \begin{exampleblock}{Overview}
  $\Sigma(\vec x_t)$ penalises the curvature of the value function. The higher the noise, the smoother the value function, i.e. $\mathrm{trace}\left(\nabla^2_{\vec{x}_t \vec{x}_t} v(\vec x_t, t)\vec \Sigma(\vec x_t)\right)$.  
  \end{exampleblock}
\end{frame}

\begin{frame}
  \frametitle{Results}
  Using value function as terminal costs for MPPI on hardware with learnt dynamics enables comparable to a \textbf{15x} longer planning horizon.
\fontsize{8pt}{10}\selectfont
\begin{center}
\movie[width=7cm,height=5.5cm,poster,showcontrols]{Click to view}{socsr.mp4}    
\end{center}
\end{frame}

\begin{frame}{Conclusion}
\begin{exampleblock}{Contribution}
\begin{itemize}
    \item Global policy
    \item Flexibility to discontinuity via noise
    \item Better efficiency
\end{itemize}
\end{exampleblock}

\begin{alertblock}{Caveats}
\begin{itemize}
    \item Requires full dynamics information  
    \item SDE rollouts are considerably more memory-intensive
\end{itemize}
\end{alertblock}
\vfill \tiny {Neural Stochastic Optimal Control (in submission)}
\end{frame}


\begin{frame}{Overall conclusion and future work}
    
\begin{exampleblock}{Conclusions}
\begin{itemize}
    \item Parametrise function spaces for generalisation
    \item Use available gradients to enable efficiency
    \item Use stochastic formulation to enable smoothing
\end{itemize}
\end{exampleblock}

\begin{alertblock}{Future work}
\begin{itemize}
    \item Integration with modular differentiable simulators
    \item Implementation of more complex dynamics  
    \item Extensible implementation to be released to the community
    \item Evaluating performance at scale
\end{itemize}
\end{alertblock}

\end{frame}


\begin{frame}
\frametitle{Acknowledgements}
\begin{center}
    Joint work with my supervisors at\\
    The University of Edinburgh
\end{center}
\begin{columns}
    \begin{column}{0.5\textwidth}
        \begin{center}
            \includegraphics[width=0.5\linewidth]{mmist.jpg} % Ensure the image is in the same directory or provide the full path
            \linebreak
            Michael Mistry
        \end{center}
    \end{column}
    \begin{column}{0.5\textwidth}
        \begin{center}
            \includegraphics[width=0.5\linewidth]{ston.png} % Ensure the image is in the same directory or provide the full path
            \linebreak
            Steve Tonneau
        \end{center}
    \end{column}
\end{columns}
\end{frame}

\end{document}